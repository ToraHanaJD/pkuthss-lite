% Copyright (c) 2014,2016 Casper Ti. Vector
% Public domain.

\chapter{致谢}
首先,我要感谢北京大学给予我学习的机会与场所。在北大的三年时光中,遇见了很多人,经历过很多事,认识到世界的宽广。这段时间谈不上一帆风顺,有迷茫,也有挫折,但也正是这迷茫与挫折,给予我重新审视自己、审视环境的契机。\\
\indent 非常感谢赵海燕老师和张伟老师。
赵老师给予我在软工所窥探科研奥秘的机会,总能认真地聆听我不成熟的想法,并提供实用的意见与建议。
张老师和赵老师在科研、写作、汇报、生活等方面给予我诸多指导,在日常科研工作中为我们指明方向。
张老师对表达的要求、对逻辑的把控、对治学的严谨、对科研的追求,令我受益匪浅。
赵老师对北大自由精神的传承和体现,让我收获颇丰。非常感谢赵老师在论文写作中提供的帮助和支持。\\
\indent 感谢潘熙师兄,在学习、工作、生活中给予我无私的帮助与鼓励。
感谢周衍师兄和渠吉超师兄,引领我踏入众多不曾涉足的领域,深刻体会到世界之大无奇不有,新燕园活动室里多个举杯共饮的下午,必是此生宝贵的财富。
感谢许多的同学,与你们的相识令最后两年校园生活愈发丰富多彩,走遍北小营村探求美食的光景,始终让人忍俊不禁。\\
\indent 感谢张明悦师兄、褚文杰师姐、罗懿行师姐、刘坤师兄、樊梦丹师姐、俞蔼伦师姐、刘伟同学、施亦凡同学、陈显彩同学、曲思睿同学以及黄博弈同学等课题组同学们在学习、工作等方面的诸多帮助。\\
\indent 感谢1726实验室里一届又一届的同学们,与你们相处的五年时光是一段愉快的经历。感谢八年时光中在28楼和42楼相遇的九位舍友,感谢你们在生活中对我的包容与帮助。\\
\indent 最后感谢我的父母,感谢父母尊重我的每一个选择,给予我无条件的支持与鼓励。远在他乡求学多年,未能给予你们足够的陪伴,对此深表歉意。在此向你们献上最诚挚的感谢。\\
\indent 有缘再会。

% vim:ts=4:sw=4
