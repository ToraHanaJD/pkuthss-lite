% Copyright (c) 2014,2016,2021 Casper Ti. Vector
% Public domain.
\begin{cabstract}
\indent 形式化验证是确保操作系统内核正确性、安全性和可靠性的重要手段。本研究基于霍尔逻辑(Hoare Logic)及其扩展——分离逻辑(Separation Logic),结合Coq证明助手,开展了一项操作系统内核形式化验证的工程实践,重点验证了内存管理、进程隔离和并发同步等关键模块的正确性。通过分离逻辑的资源分离特性,我们清晰地描述并验证了内核模块的资源使用情况,避免了资源冲突和竞争条件。针对并发环境的复杂性,我们采用并发分离逻辑(Concurrent Separation Logic),确保了多线程环境下资源共享与同步机制的正确性。借助Coq的交互式证明机制,我们从高层规约逐步精化并验证了内核实现,确保其与规约的一致性。\\
\indent 本文首先分析了操作系统形式化验证的研究现状,总结了定理证明、模型检测和符号执行等方法在操作系统验证中的应用场景及其局限性。随后,详细介绍了霍尔逻辑和分离逻辑的基本原理及其在形式化验证中的优势,特别是分离逻辑在描述动态内存分配和并发资源共享方面的表达能力。在此基础上,我们基于工程实践,利用分离逻辑对操作系统内核的关键功能模块进行了形式化规约与验证,重点验证了内存管理和并发同步机制的正确性。\\
\indent 在验证过程中,我们充分利用分离逻辑的资源分离特性,清晰地描述和验证了内核中不同模块的资源使用情况,避免了资源竞争和冲突。针对并发环境下的复杂性,我们引入了并发分离逻辑(Concurrent Separation Logic),验证了多线程环境下的资源共享与同步机制的正确性。通过Coq证明助手的交互式证明机制,我们逐步完成了从高层规约到低层实现的精化验证,确保了内核代码与规约的一致性。\\
\indent 本研究的贡献主要体现在以下几个方面:
\begin{itemize}
    \item 基于分离逻辑和Coq证明助手,实现了操作系统内核关键模块的形式化验证,为高可信操作系统的工程实践提供了参考。
    \item 验证了分离逻辑在操作系统形式化验证中的实用性,特别是在内存管理和并发同步方面的优势。
    \item 通过工程实践,展示了形式化验证在操作系统验证和开发中的实际应用价值,为后续研究提供了经验。
\end{itemize}
\indent 本研究为操作系统形式化验证的工程实践提供了具体案例,对推动高可信操作系统的开发具有重要意义。
\end{cabstract}
\begin{eabstract}
\indent Formal verification of operating systems (OS) is a critical approach to ensuring the correctness, security, and reliability of OS kernels. Traditional testing methods often fall short of comprehensively addressing potential errors in complex systems, while formal verification employs mathematical techniques to rigorously prove that a system adheres to its specifications. This study focuses on an engineering practice of formal verification for OS kernels, utilizing Hoare Logic and its extension, Separation Logic, along with the Coq proof assistant. The practice specifically targets the verification of key kernel modules, including memory management, process isolation, and concurrent synchronization mechanisms.\\
\indent The paper begins by reviewing the state of the art in OS formal verification, summarizing the applications and limitations of theorem proving, model checking, and symbolic execution in this domain. It then introduces the foundational principles of Hoare Logic and Separation Logic, highlighting the latter's strengths in describing dynamic memory allocation and shared resource management in concurrent environments. Building on these principles, the study conducts formal specification and verification of critical OS kernel modules based on practical engineering requirements.\\
\indent In this engineering practice, Separation Logic's resource separation properties are leveraged to clearly describe and verify resource usage across different kernel modules, preventing resource conflicts and race conditions. For handling concurrency complexities, Concurrent Separation Logic is employed to verify the correctness of resource sharing and synchronization mechanisms in multi-threaded environments. Using Coq's interactive proof mechanisms, the practice incrementally refines and verifies the kernel implementation against its high-level specifications, ensuring consistency between the code and its formal model.\\
\indent The contributions of this study are mainly reflected in the following aspects:
\begin{itemize}
    \item formal verification of critical OS modules using Separation Logic and Coq, providing a practical reference for high-assurance OS development.
    \item a demonstration of Separation Logic's effectiveness in memory management and concurrency.
    \item a real-world case study highlighting the value of formal verification in OS engineering.
\end{itemize}
\indent This work contributes to advancing reliable OS development and lays the groundwork for future research on scalable and automated verification techniques.
\end{eabstract}

% vim:ts=4:sw=4