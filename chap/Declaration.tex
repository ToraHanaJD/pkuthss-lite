{
    \ctexset{section = {
        format+ = {\centering}, beforeskip = {40bp}, afterskip = {15bp}
    }}
    \specialchap{北京大学学位论文原创性声明和使用授权说明}

    % 学校书面要求本页面不要页码,但在给出的 Word 模版中又有页码。
    % 此处以学校书面要求为准。
    \thispagestyle{empty}
    \mbox{}\vspace*{-3em}
    \section*{原创性声明}

    本人郑重声明:
    所呈交的学位论文,是本人在导师的指导下,独立进行研究工作所取得的成果。
    除文中已经注明引用的内容外,
    本论文不含任何其他个人或集体已经发表或撰写过的作品或成果。
    对本文的研究做出重要贡献的个人和集体,均已在文中以明确方式标明。
    本声明的法律结果由本人承担。
    \vskip 1em
    \rightline{%
        论文作者签名:\hspace{5em}%
        日期:\hspace{2em}年\hspace{2em}月\hspace{2em}日%
    }

    \section*{%
        学位论文使用授权说明\\[-0.33em]
        \textmd{\zihao{5}(必须装订在提交学校图书馆的印刷本)}%
    }

    本人完全了解北京大学关于收集、保存、使用学位论文的规定,即:
    \begin{itemize}
        \item 按照学校要求提交学位论文的印刷本和电子版本;
        \item 学校有权保存学位论文的印刷本和电子版,
            并提供目录检索与阅览服务,在校园网上提供服务;
        \item 学校可以采用影印、缩印、数字化或其它复制手段保存论文;
        \item 因某种特殊原因须要延迟发布学位论文电子版,
            授权学校 $\Box$\nobreakspace{}一年 /
            $\Box$\nobreakspace{}两年 /
            $\Box$\nobreakspace{}三年以后,在校园网上全文发布。
    \end{itemize}
    \centerline{(保密论文在解密后遵守此规定)}
    \vskip 1em
    \rightline{%
        论文作者签名:\hspace{5em}导师签名:\hspace{5em}%
        日期:\hspace{2em}年\hspace{2em}月\hspace{2em}日%
    }

    % 若须排版二维码,请将二维码图片重命名为“barcode”,
    % 转为合适的图片格式,放在 fig 目录下,然后去掉下面 2 行的注释。
    % \vfill\noindent
    % \includegraphics[height = 5em]{fig/barcode}
}
