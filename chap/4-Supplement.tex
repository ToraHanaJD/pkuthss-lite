\chapter{第四章}
本章介绍一些模板功能之外的常见需求的实现方法。
 
\section{本章小结} \label{sec:chap3章节小结}
% TODO
% 数据结构和算法的相辅相成
% 增量融合离不开中间结果存储
本章主要介绍了一种面向知识演化的多图融合机制。该机制通过一个基于多路归并思想的树型结构——图融合树,来维护输入的个体知识图谱、中间融合结果以及最终融合结果。在图融合树的构建中,使用一种自底向上的建树方法,根据树节点相应的知识图谱的关联度,通过聚类算法不断对树节点进行分组并构建新的树节点。在图融合树的更新中,自下而上地更新叶子节点到根节点的路径中所有节点相应的知识图谱。\\
\indent 随后,本章介绍了一种基于广义熵的图融合算法。首先介绍了一种基于广义熵的融合质量度量方法,并设计了一种知识图谱节点相似度的度量方法。然后介绍了图融合算法中的全量融合以及增量融合算法。全量融合算法用于图融合树构建时,对一组知识图谱的所有节点进行对齐并融合。增量融合算法用于知识图谱增量更新时,对部分的知识图谱节点进行重新对齐,以更新已有的融合知识图谱。最后介绍了基于词向量相似度和基于字符串编辑距离的知识图谱节点语义相似度的度量方式。